\documentclass[12pt]{article}
\begin{document}

separate choosing U into two steps:

choosing the rows orientation-wise
choosing the magnitude (+/-) of the rows

\begin{bmatrix}
1 & 0 \\
0 & 1\\
1 & 1\\
\end{bmatrix}

this is a non-degerate case with 6 possible patterns

+++,++-,---,--+,+- ,-+

(slight deformity to extremes because we removed two intermediate sign patterns)



+^n,+^{n-1}-, ...., -+^{n-1}

non-degeracy = any r by r subset of u has no 


U * w -> sign pattern
and you want arb. sp v

sign(diag(sign(w/v)) * U * w) = v


separate choosing U into two steps:

step 1: choosing the rows as one-dimensional subspaces (spans)
step 2: choosing the magnitude (+/-) of the rows

the distribution of (the set of sign patterns) with respect to step 2
is dependent on step 1


claim: so long as U is non-degenerate (all r x r submatrices have rank r),
the distribution of (the set of sign patterns) with respect to step 2
is constant

so if a metric (like avg. hamming distance between two sign patterns) is constant w.r.t. step 2, it's just constant full stop.
you have to exhaustively get every possible sign pattern for this



\end{document}