\documentclass[12pt]{article}
\begin{document}

Coherence is a general term 


The Babel function is an extension of mututal coherence, where you take the largest clump of p values such that they are all very close to one direction. 
Neither of these seem great... they're clearly working around identifying a small set of biased values.

I feel like we want to identify general clustering. Here, we can try data clustering on the surface of the r-ball.
first thought: rectangular clustering metrics on the spherical graph. Unsure how well angle || distance translates to actual vector angles...

Found: Clustering tendency
The "hopkins statistic" is a method of gauging clustering tendency of a group of data points:
	-Sample m data points
	-Generate m uniformly rand. distributed data points.
	-take the closest pairwise distance of the rand and divide it by the closest pairwise distance between the data points and the rand points

The statistic should be about .5 when the rows are uniformly distributed, and 

My plan is to adapt this to work on an n-ball surface as follows:
I will use the "distance" metric equal to sin(theta), where theta is the distance between the angles.




\end{document}